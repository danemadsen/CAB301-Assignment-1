\documentclass[12pt,a4paper]{article}
\usepackage[utf8]{inputenc}
\usepackage{graphicx}

\begin{document}
	\begin{titlepage}
		
		\begin{center}
			\includegraphics[width=0.5\textwidth]{QUT.jpg}\\
			[0.03\textheight]  
			\Large\textbf{Bachelor of IT (Computer Science)}\\
			\Large\textbf{Assignment Title}\\
			\large\textbf{Course}\\
			[0.02\textheight]
			\large\textsl{Dane Madsen}\\
			\large\textsl{n10983864@qut.edu.au}
		\end{center}
		
	\end{titlepage}
	\tableofcontents
	\newpage
	
	\section{Algorithm Design}
		\subsection{Jobs ADT}
			\subsubsection{IsValidId Method}
				This method checks whether a provided job ID is valid. It achieves this 
				by checking that the provided ID is greater than the minimum valid ID (1) and 
				less than the maximum valid ID (999). If the ID is meets these criteria, the 
				method returns true indicating the ID is valid, otherwise it returns false 
				indicating the ID is invalid.\\
				
				\textbf{ALGORITHM} \textit{IsValidId(v)}\\
				\null\qquad\quad// Given an integer (\textit{v})\\
				\null\qquad\quad// Returns True if \textit{v} is a valid job ID\\
				\null\qquad\quad// Otherwise returns False\\
				\null\qquad\quad\textbf{if} \textit{$v \geq 1$} \textbf{and} \textit{$v \leq 999$}\\
				\null\qquad\qquad\textbf{return} \textit{True}\\
				\null\qquad\quad\textbf{else}\\
				\null\qquad\qquad\textbf{return} \textit{False}\\

			\subsubsection{IsValidExecutionTime Method}
				This method simply checks whether a provided job execution time is valid. It achieves 
				this by simply checking whether the execution time is greater than 0. If the execution 
				time is greater than 0, the method returns true indicating the execution time is valid,
				otherwise it returns false indicating the execution time is invalid.\\

				\textbf{ALGORITHM} \textit{IsValidExecutionTime(v)}\\
				\null\qquad\quad// Given an integer (\textit{v})\\
				\null\qquad\quad// Returns True if \textit{v} is a valid job execution time\\
				\null\qquad\quad// Otherwise returns False\\
				\null\qquad\quad\textbf{if} \textit{$v > 0$}\\
				\null\qquad\qquad\textbf{return} \textit{True}\\
				\null\qquad\quad\textbf{else}\\
				\null\qquad\qquad\textbf{return} \textit{False}

			\newpage

			\subsubsection{IsValidPriority Method}
				This method checks whether a provided job priority is valid. It achieves this by 
				checking that the provided priority is greater than or equal to the minimum valid priority (1) and 
				less than or equal to the maximum valid priority (9). If the priority is meets these criteria, the 
				method returns true indicating the priority is valid, otherwise it returns false 
				indicating the priority is invalid.\\

				\textbf{ALGORITHM} \textit{IsValidPriority(v)}\\
				\null\qquad\quad// Given an integer (\textit{v})\\
				\null\qquad\quad// Returns True if \textit{v} is a valid job priority\\
				\null\qquad\quad// Otherwise returns False\\
				\null\qquad\quad\textbf{if} \textit{$v \geq 1$} \textbf{and} \textit{$v \leq 9$}\\
				\null\qquad\qquad\textbf{return} \textit{True}\\
				\null\qquad\quad\textbf{else}\\
				\null\qquad\qquad\textbf{return} \textit{False}\\
			
			\subsubsection{IsTimeReceived Method}
				This method checks whether a provided job time received is valid. It achieves this by
				checking that the provided time received is greater than zero. If the time received is
				greater than zero, the method returns true indicating the time received is valid, otherwise
				it returns false indicating the time received is invalid.\\

				\textbf{ALGORITHM} \textit{IsTimeReceived(v)}\\
				\null\qquad\quad// Given a job time received (\textit{v})\\
				\null\qquad\quad// Returns True if \textit{v} is a valid time received\\
				\null\qquad\quad// Otherwise returns False\\
				\null\qquad\quad\textbf{if} \textit{$v > 0$}\\
				\null\qquad\qquad\textbf{return} \textit{True}\\
				\null\qquad\quad\textbf{else}\\
				\null\qquad\qquad\textbf{return} \textit{False}
		
		\newpage
			
		\subsection{JobCollection ADT}
			\subsubsection{Add Method}
				This method adds a job to the job collection. It achieves this by first checking that the job 
				doesn't already exist in the collection. If the job does already exist in the collection, the 
				method returns false indicating the job was not added to the collection. If the job does not 
				already exist in the collection, the method adds the job to the collection, increments the count 
				variable and returns true.\\
				
				\textbf{ALGORITHM} \textit{Add(v)}\\
				\null\qquad\quad// Let (\textit{n}) be count\\
				\null\qquad\quad// Given a job (\textit{v})\\
				\null\qquad\quad// Returns True if \textit{v} was added to the jobs array (\textit{J})\\
				\null\qquad\quad// Otherwise returns False\\
				\null\qquad\quad\textbf{for} \textit{$i \gets 0$} \textbf{in} \textit{n - 1} \textbf{do}\\
				\null\qquad\qquad\textbf{if} \textit{$v.id = J[i].id$}\\
				\null\qquad\qquad\quad\textbf{return} \textit{False}\\
				\null\qquad\quad\textbf{else}\\
				\null\qquad\qquad\textit{$J[n] \gets v$}\\
				\null\qquad\qquad\textit{$n \gets n + 1$}\\
				\null\qquad\qquad\textbf{return} \textit{True}\\

			\subsubsection{Contains Method}
				This method is used to check if a job exists in the job collection. It achieves this by 
				checking if there is a job in the collection with the same ID as the provided job ID. If there is 
				a job with the same ID the method returns true, otherwise it returns false.\\

				\textbf{ALGORITHM} \textit{Contains(v)}\\
				\null\qquad\quad// Let (\textit{n}) be count\\
				\null\qquad\quad// Given an integer (\textit{v})\\
				\null\qquad\quad// Returns True if a job with the ID \textit{v} exists in the jobs array (\textit{J})\\
				\null\qquad\quad// Otherwise returns False\\
				\null\qquad\quad\textbf{for} \textit{$i \gets 0$} \textbf{in} \textit{n - 1} \textbf{do}\\
				\null\qquad\qquad\textbf{if} \textit{$v = J[i].id$}\\
				\null\qquad\qquad\quad\textbf{return} \textit{True}\\
				\null\qquad\quad\textbf{else}\\
				\null\qquad\qquad\textbf{return} \textit{False}
			
			\newpage

			\subsubsection{Find Method}
				This method is used to find a job in the job collection. It achieves this by using the provided 
				job ID to find the job in the collection. If the job is found, the method returns the job, 
				otherwise it returns null.\\

				\textbf{ALGORITHM} \textit{Find(v)}\\
				\null\qquad\quad// Let (\textit{n}) be count\\
				\null\qquad\quad// Given an integer (\textit{v})\\
				\null\qquad\quad// Returns the job with the ID \textit{v} if it exists in the jobs array (\textit{J})\\
				\null\qquad\quad// Otherwise returns null\\
				\null\qquad\quad\textbf{for} \textit{$i \gets 0$} \textbf{in} \textit{n - 1} \textbf{do}\\
				\null\qquad\qquad\textbf{if} \textit{$v = J[i].id$}\\
				\null\qquad\qquad\quad\textbf{return} \textit{J[i]}\\
				\null\qquad\quad\textbf{return} \textit{null}\\

			\subsubsection{Remove Method}
				This method is used to remove a job from the job collection. It achieves this by using the 
				provided job ID to find the job in the collection. If the job is found, the method removes the 
				job from the collection, decrements the count variable and returns true. If the job is not found, 
				the method returns false.\\

				\textbf{ALGORITHM} \textit{Remove(v)}\\
				\null\qquad\quad// Let (\textit{n}) be count\\
				\null\qquad\quad// Given an integer (\textit{v})\\
				\null\qquad\quad// Returns True if a job with the ID \textit{v} was removed\\
				\null\qquad\quad// from the jobs array (\textit{J})\\
				\null\qquad\quad// Otherwise returns False\\
				\null\qquad\quad\textbf{for} \textit{$i \gets 0$} \textbf{in} \textit{n - 1} \textbf{do}\\
				\null\qquad\qquad\textbf{if} \textit{$v = J[i].id$}\\
				\null\qquad\qquad\qquad\textbf{for} \textit{$j \gets 0$} \textbf{in} \textit{n - 2} \textbf{do}\\
				\null\qquad\qquad\qquad\quad\textit{$J[j] \gets J[j + 1]$}\\
				\null\qquad\qquad\qquad\textit{$n \gets n - 1$}\\
				\null\qquad\qquad\qquad\textbf{return} \textit{True}\\
				\null\qquad\quad\textbf{return} \textit{False}\\
			
			\newpage
			
			\subsubsection{ToArray Method}
				This method is used to convert the job collection to an array. It achieves this by creating a 
				new array of the same size as the job collection and then copying the jobs from the job 
				collection to the new array. The method then returns the new array.\\

				\textbf{ALGORITHM} \textit{ToArray()}\\
				\null\qquad\quad// Let (\textit{n}) be count\\
				\null\qquad\quad// Returns a new array of copied from the jobs array (\textit{J})\\
				\null\qquad\quad\textit{$A \gets new Job[n]$}\\
				\null\qquad\quad\textbf{for} \textit{$i \gets 0$} \textbf{in} \textit{n - 1} \textbf{do}\\
				\null\qquad\qquad\textit{$A[i] \gets J[i]$}\\
				\null\qquad\quad\textbf{return} \textit{A}\\
				
		\subsection{Scheduler ADT}
	
	\section{Analysis}
		\subsection{Jobs ADT}
		\subsection{JobCollection ADT}
		\subsection{Scheduler ADT}

	\section{Testing}
		\subsection{Jobs ADT}
		\subsection{JobCollection ADT}
		\subsection{Scheduler ADT}

\end{document}